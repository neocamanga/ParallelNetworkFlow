\section{Instruções Gerais}

A quantidade máxima recomendada de páginas para o artigo é de 20 páginas sem anexos ou apêndices.
Quando escrever o seu artigo, por favor atente às seguintes instruções:

\subsection{Resumo e abstract}
Os artigos escritos em língua portuguesa devem ter também o resumo e as palavras-chave traduzidos para a língua inglesa.

\subsection{Seções e subseções}
As seções devem ser numeradas com algarismos romanos e ter o título centralizado. Já as subseções devem ser numeradas com letras maiúsculas e ter o título justificado, caso haja sequência de subtítulos as letras devem ser minúsculas e justificadas. 

\subsubsection{Sub-subseção}

Exemplo de uma sub-subseção.

\subsection{Listas}

Exemplos de como se pode fazer itens com itemize

\begin{itemize}
\item Item 1.
\item Item 2.
\item Item 3.
\end{itemize}

e com enumerate

\begin{enumerate}
 \item enumA
 \item enumB
\end{enumerate}

\subsection{Figuras e Tabelas}

Figuras e Tabelas devem ser incluídas como parte do texto sempre que possível, caso contrário, agrupe-as ao final do texto. As Figuras deve ter seus rótulos posicionados depois das mesmas, com alinhamento centralizado. A sua numeração deve ser feita com algarismos arábicos. Para as Tabelas, o procedimento é diferente: seus rótulos devem ser posicionados antes das mesmas, centralizados, e a numeração deve ser feita com algarismos romanos. As figuras devem ser referenciadas no texto, da seguinte forma: Na Figura \ref{fig:unoesc} é apresentado a logo da UNOESC.

\begin{figure}[h]
    \centering
    \includegraphics{figuras/unoesc.jpg}
    \caption{Uma figura. O título deve ser colocado abaixo da mesma.}
    \label{fig:unoesc}
\end{figure}

\subsection{Equações}
A numeração das equações deve ser entre parênteses e alinhada à direita. O cálculo é feito por meio de (\ref{eq:mgf}).

\begin{equation}
\label{eq:mgf}
\phi_X(s)=E[e^{sx}]
\end{equation}

%Consulte a página 
Para mais símbolos matemáticos, consulte o LaTeX wiki \cite{latexMath}

\subsection{Fontes}

Use fonte do tipo Times New Roman ou similar. Os tamanhos a serem usados são mostrados na Tabela \ref{tab:tabela1}

\begin{table}[h]
\centering
\caption{Tamanhos e Tipos de Letras}
\label{tab:tabela1}
\begin{adjustbox}{max width=\textwidth}
\begin{tabular}{lcr}
\hline
TEXTO                     & TAMANHO & ESTILO          \\ \hline
Título                    & 24pt    & Negrito         \\
Nome do autor             & 11pt    & Normal          \\
Afiliação                 & 10pt    & Normal          \\
Texto principal           & 10pt    & Normal          \\
Título das seções         & 10pt    & Caixa Alta      \\
Título das subseções      & 10pt    & Itálico         \\
Título do resumo/abstract & 9pt     & Negrito,Itálico \\
Resumo/Abstract           & 9pt     & Negrito         \\
Título das figuras        & 8pt     & Normal          \\
Título das tabelas        & 8pt     & Caixa Alta      \\
Texto das tabelas         & 8pt     & Normal          \\
Referências               & 8pt     & Normal          \\ \hline
\end{tabular}
\end{adjustbox}
\end{table}


%Se desejar, consulte a página \url{https://www.tablesgenerator.com} para criar a estrutura da sua tabela em latex. Esta página fornece uma interface gráfica que facilita a geração de tabelas em latex. Depois de gerado o código fonte pela página, insira este código no seu texto e faça os ajustes necessários.

\subsection{References}

Liste as referências em ordem numérica ao final do artigo. Ao final deste texto tem-se vários exemplos de como listá-las, dependendo do tipo. Denote as citações dentro do texto através de colchetes (por exemplo \cite{LIMA2018}). Ao referenciar mais de um trabalho, use o mesmo par de colchetes, como exemplo: \cite{LUCKMANN2008,livro-unoesc,NR10,abntex2modelo}. 

Segue um exemplo para citações textuais: ``De acordo com \textcite{CHAPMAN2013}'' 

\subsection{Outras questões}
Não use notas de rodapé a menos que sejam estritamente necessárias; neste caso, procure não agrupá-las. 