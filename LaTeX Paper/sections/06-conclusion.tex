\section{Conclusion}
% Network flow is pretty cool. By parallelizing them we advanced human civilization to a new frontier, if you will, of human knowledge. We have now reached the point of singularity and we are infinite beings. With this paper, this is what we now offer to society, to humans, and to the universe. (Conclusion still pending)

Overall, an important lesson that was gleaned from this project was the potential misapplication of parallel programming. Not every algorithm benefits from concurrency the same way; in the case with some of Dinic's potential applications with different graph traversal algorithms, DFS suffered more of a performance overhead than actual gains in speed. Ford-Fulkerson's parallel implementation never quite sees significant practical usage until test-cases began to show considerable speedups from larger graph sizes. Edmonds-Karp was never able to see any valuable reason for parallelization, but was able to improve upon its initial approach. We gained a significant amount of knowledge in the field of network flow through this research as well as some insight into how locking can affect the run times of algorithms in parallel. 